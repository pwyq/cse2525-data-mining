%%%%%%%%%%%%%%%%%%%%%%%%%%%%%%%%%%%%%%%%%%%%%%%%%%%%%%%%%%%%%%%%%%%%%%%%%%%%%%%%
%% ************************************************************************** %%
%% *                                Settings                                * %%
%% ************************************************************************** %%
%%%%%%%%%%%%%%%%%%%%%%%%%%%%%%%%%%%%%%%%%%%%%%%%%%%%%%%%%%%%%%%%%%%%%%%%%%%%%%%%
\documentclass{ece}
\loadglsentries{gls}
\usepackage{tabto}
\usepackage{tabularx}
\glsaddall
\addbibresource{reference}
\usepackage[flushleft]{threeparttable}
\usepackage[autostyle=false, style=english]{csquotes}
\MakeOuterQuote{"}


%%%%%%%%%%%%%%%%%%%%%%%%%%%%%%%%%%%%%%%%%%%%%%%%%%%%%%%%%%%%%%%%%%%%%%%%%%%%%%%%
% Make sure the following block contains the correct information               %
%%%%%%%%%%%%%%%%%%%%%%%%%%%%%%%%%%%%%%%%%%%%%%%%%%%%%%%%%%%%%%%%%%%%%%%%%%%%%%%%
\reporttitle{Netflix Challenge: Movie Rating Prediction}
\employername{CSE-2525 Data Mining}
\employerstreetaddress{Thomas Abeel, Gosia Migut}
\authorname{Yanqing Wu}
\studentnumber{5142571}
\userid{yanqingwutudelft}
\program{Exchanged Computer Engineering}
%%%%%%%%%%%%%%%%%%%%%%%%%%%%%%%%%%%%%%%%%%%%%%%%%%%%%%%%%%%%%%%%%%%%%%%%%%%%%%%%
% end of information block...                                                  %
%%%%%%%%%%%%%%%%%%%%%%%%%%%%%%%%%%%%%%%%%%%%%%%%%%%%%%%%%%%%%%%%%%%%%%%%%%%%%%%%

\begin{document}

%%%%%%%%%%%%%%%%%%%%%%%%%%%%%%%%%%%%%%%%%%%%%%%%%%%%%%%%%%%%%%%%%%%%%%%%%%%%%%%%
%% ************************************************************************** %%
%% *                               Title Page                               * %%
%% ************************************************************************** %%
%%%%%%%%%%%%%%%%%%%%%%%%%%%%%%%%%%%%%%%%%%%%%%%%%%%%%%%%%%%%%%%%%%%%%%%%%%%%%%%%

\maketitle

%%%%%%%%%%%%%%%%%%%%%%%%%%%%%%%%%%%%%%%%%%%%%%%%%%%%%%%%%%%%%%%%%%%%%%%%%%%%%%%%
%% ************************************************************************** %%
%% *                           Table of Contents                            * %%
%% ************************************************************************** %%
%%%%%%%%%%%%%%%%%%%%%%%%%%%%%%%%%%%%%%%%%%%%%%%%%%%%%%%%%%%%%%%%%%%%%%%%%%%%%%%%

\tableofcontents

%%%%%%%%%%%%%%%%%%%%%%%%%%%%%%%%%%%%%%%%%%%%%%%%%%%%%%%%%%%%%%%%%%%%%%%%%%%%%%%%
%% ************************************************************************** %%
%% *                            List of Figures                             * %%
%% ************************************************************************** %%
%%%%%%%%%%%%%%%%%%%%%%%%%%%%%%%%%%%%%%%%%%%%%%%%%%%%%%%%%%%%%%%%%%%%%%%%%%%%%%%%

\listoffigures

%%%%%%%%%%%%%%%%%%%%%%%%%%%%%%%%%%%%%%%%%%%%%%%%%%%%%%%%%%%%%%%%%%%%%%%%%%%%%%%%
%% ************************************************************************** %%
%% *                             List of Tables                             * %%
%% ************************************************************************** %%
%%%%%%%%%%%%%%%%%%%%%%%%%%%%%%%%%%%%%%%%%%%%%%%%%%%%%%%%%%%%%%%%%%%%%%%%%%%%%%%%

\listoftables

%%%%%%%%%%%%%%%%%%%%%%%%%%%%%%%%%%%%%%%%%%%%%%%%%%%%%%%%%%%%%%%%%%%%%%%%%%%%%%%%
%% ************************************************************************** %%
%% *                                  Body                                  * %%
%% ************************************************************************** %%
%%%%%%%%%%%%%%%%%%%%%%%%%%%%%%%%%%%%%%%%%%%%%%%%%%%%%%%%%%%%%%%%%%%%%%%%%%%%%%%%

\body

\section{Introduction}

The report, entitled ``Netflix Challenge: Movie Rating Prediction'', is prepared as my Challenge report for the course CSE2525-Data Mining at the Technische Universiteit Delft.
The purpose of this report is to develop a recommendation system for predicting movie ratings.
The goal of the recommendation system is to achieve Root Mean Square Error (RMSE) as small as possible on an unseen dataset.

\subsection{Netflix Datasets}

\begin{table}[ht!]
    \caption[The Basic Information of Provided Data Sets]{The Basic Information of Provided Data Sets}	
    \label{tab:table-1}	
    \centering
    \begin{tabular*}{\textwidth}{@{\extracolsep{\fill}}lllllS[table-format=5.2]}	
    	% l=left-justified column, c=centered column, r=right-justified column
        % 4.4 means 4 digits before and after the decimal point	
        \hline	
        \textbf{Dataset} 
        	& \multicolumn{1}{c}{\textbf{Features}} & \multicolumn{1}{c}{\textbf{Mean}} 
            & \multicolumn{1}{c}{\textbf{Std}} & \multicolumn{1}{c}{\textbf{Min \& Max}} \\	
        \hline	
        % ~ gives space	
        users
            & gender
            & 0.72
            & 0.45
        	& \begin{tabular}{@{}l@{}} 0.00 \\ 1.00 \end{tabular}
\\
        	& age
        	& 30.64
        	& 12.90
        	& \begin{tabular}{@{}l@{}} 1.00 \\ 56.00 \end{tabular}
\\
        	& profession
        	& 8.15
        	& 6.33
        	& \begin{tabular}{@{}l@{}} 0.00 \\ 20.00 \end{tabular}
\\      movies
        	& year
        	& 1985.81
        	& 16.91
        	& \begin{tabular}{@{}l@{}} 1919.00 \\ 2000.00 \end{tabular}
\\
			& title (string)
			& -
			& -
			& -
\\		ratings
			& rating
			& 3.58
			& 1.12
			& \begin{tabular}{@{}l@{}} 1.00 \\ 5.00 \end{tabular}
\\      \hline	
   \end{tabular*}
   \begin{tablenotes}
     \item In users - `gender', `0' and `1' indicates female users and male users, respectively;
     \item In movies - `year', only non-zero entries are considered.
   \end{tablenotes}
\end{table}

Three datasets are provided for training, as described in \Cref{tab:table-1}.
There is a total of 910,190 ratings, which were given by 6,040 users and 3,706 movies.
The rest `predictions.csv' file is used for final testing, which contains only `userID` and `movieID' for each entry.


\section{Methodology}

Data interpretation was first conducted at the beginning of the Challenge, as demonstrated in \Cref{tab:table-1}.
After examining the data and reviewing the course materials, the first version of the recommendation algorithm was decided as Collaborative Filtering (CF).
More specifically, Item-Item CF was selected as the first attempt to the Challenge.
Item-Item CF is more reliable than User-User CF in practice \cite{mmds}, because items are much less dynamic than users.

\subsection{Item-Item Collaborative Filtering}

\begin{equation}
	\begin{aligned}\label{eq:1}
& r_{xi} = b_{xi} + \frac{\sum\limits_{j\in N(i;x)} S_{ij} \cdot (r_{xj} - b_{xj}) }{\sum\limits_{j\in N(i;x)} S_{ij}}  \\
& b_{xi} = \mu + b_x + b_i
	\end{aligned}
\end{equation}

Equation \eqref{eq:1} shows the core idea of the algorithm.
Term $r_{xi}$ represents the rating of user $x$ on movie $i$.
Term $s_{ij}$ is the similarity of movie $i$ and movie $j$, which is measured by cosine similarity.
Term $b_{xi}$ is the baseline estimator for $r_{xi}$, where $\mu$ is the average ratings of all movies, $b_x$ is the rating deviation of user $x$ and $b_i$ is the rating deviation of movie $i$.

\section{Results}

Some more text.

\section{Discussion}

asdfasdfasdf

\section{Conclusions and Future Work}

From the analysis in the report body, it was concluded that...

%%%%%%%%%%%%%%%%%%%%%%%%%%%%%%%%%%%%%%%%%%%%%%%%%%%%%%%%%%%%%%%%%%%%%%%%%%%%%%%%
%% ************************************************************************** %%
%% *                               References                               * %%
%% ************************************************************************** %%
%%%%%%%%%%%%%%%%%%%%%%%%%%%%%%%%%%%%%%%%%%%%%%%%%%%%%%%%%%%%%%%%%%%%%%%%%%%%%%%%

\printbibliography[heading=none]

%%%%%%%%%%%%%%%%%%%%%%%%%%%%%%%%%%%%%%%%%%%%%%%%%%%%%%%%%%%%%%%%%%%%%%%%%%%%%%%%
%% ************************************************************************** %%
%% *                               Appendices                               * %%
%% ************************************************************************** %%
%%%%%%%%%%%%%%%%%%%%%%%%%%%%%%%%%%%%%%%%%%%%%%%%%%%%%%%%%%%%%%%%%%%%%%%%%%%%%%%%

% appendices use section and subsection numbering
\appendix

\section{Title of First Appendix}
\label{app:firstappx}
Use the No Spacing style.

\section{Another Appendix}
\label{app:anotherappx}
Again, use the no spacing style for appendices.

\end{document}
